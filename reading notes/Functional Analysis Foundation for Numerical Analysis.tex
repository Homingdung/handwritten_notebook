\documentclass{article}

\usepackage[margin=1in]{geometry} 
\usepackage{amsmath,amsthm,amssymb,hyperref}

\newcommand{\R}{\mathbf{R}}  
\newcommand{\Z}{\mathbf{Z}}
\newcommand{\N}{\mathbf{N}}
\newcommand{\Q}{\mathbf{Q}}

\theoremstyle{definition}
\newtheorem{definition}{Definition}[section]
\newtheorem{corollary}{Corollary}[section]
\newtheorem{question}{Question}[section]
\newtheorem{problem}{Problem}[section]
\newtheorem{theorem}{Theorem}[section]
\newtheorem{lemma}[theorem]{Lemma}
\newenvironment{solution}{\begin{proof}[Solution]}{\end{proof}}



\begin{document}

% ------------------------------------------ %
%                 START HERE                  %
% ------------------------------------------ %

\title{Functional Analysis Foundation for Numerical Analysis} % Replace with appropriate title
\author{Mingdong He\\The University of Nottingham} 

\maketitle

\tableofcontents
% -----------------------------------------------------
% The following two environments (theorem, proof) are
% where you will enter the statement and proof of your
% first problem for this assignment.
%
% In the theorem environment, you can replace the word
% "theorem" in the \begin and \end commands with
% "exercise", "problem", "lemma", etc., depending on
% what you are submitting. 
% -----------------------------------------------------
\section{Introduction}
This is my notes based on \textit{Theoretical Numerical Analysis: A Functional Analysis Framework, Kendall Atkinson, Weimin Hall}. 
\section{Linear Space}
\subsection{Algebraic Structure of Linear Space}
\subsection{Topological Structure of Linear Space}

\end{document}