\documentclass{article}
%
\usepackage[margin=1in]{geometry} 
\usepackage{amsmath,amsthm,amssymb,hyperref}
%
\newcommand{\R}{\mathbf{R}}  
\newcommand{\Z}{\mathbf{Z}}
\newcommand{\N}{\mathbf{N}}
\newcommand{\Q}{\mathbf{Q}}
%
\theoremstyle{definition}
\newtheorem{definition}{Definition}[section]
\newtheorem{corollary}{Corollary}[section]
\newtheorem{question}{Question}[section]
\newtheorem{problem}{Problem}[section]
\newtheorem{theorem}{Theorem}[section]
\newtheorem{lemma}[theorem]{Lemma}
\newenvironment{solution}{\begin{proof}[Solution]}{\end{proof}}
%
%
%
\begin{document}

% ------------------------------------------ %
%                 START HERE                  %
% ------------------------------------------ %

\title{Krylov Subspace Method} % Replace with appropriate title
\author{Mingdong He\\The University of Nottingham} 

\maketitle

\tableofcontents
% -----------------------------------------------------
% The following two environments (theorem, proof) are
% where you will enter the statement and proof of your
% first problem for this assignment.
%
% In the theorem environment, you can replace the word
% "theorem" in the \begin and \end commands with
% "exercise", "problem", "lemma", etc., depending on
% what you are submitting. 
% -----------------------------------------------------
\section{Introduction}
\begin{definition}[Krylov Subspace]
Let $A \in R^{n\times n}$ and $b \in R^n$. Then the Krylov subspace $K_j(A,b)$ is defined as:
\begin{equation}
K_j(A,b)=span\{b,Ab,A^2b,...,A^{j-1}b\} \in R^n
\end{equation}
\end{definition}
We could prove that
\begin{equation}
K_1(A,b) \subseteq K_2(A,b) \subseteq \ldots \subseteq K_j(A,b)
\end{equation}

\begin{definition}[Krylov Matrix]
Krylov Matrix is defined by, $K_j(A,b)=[b,Ab,\ldots,A^{j-1}b]$. 
\end{definition}

Each column can be computed iteratively, e.g. $Ab, A^2b=A(Ab),A^3b=A(A(Ab))$.

\section{Motivation}
\begin{theorem}[Cayley-Hamilton Theorem]
A is a $n\times n$ matrix, the characteristic polynomial has the form $p(\lambda)=\det(\lambda I_n-A)$, then p(A)=0.
\end{theorem}

Consider a matrix $A=\begin{bmatrix}0 & 2 & 1 \\ -1 & 3 & 1\\ -2 & 2 & 3\end{bmatrix}$, the characteristic polynomial of $A$ is $p(\lambda)=\lambda^3-6\lambda^2+11\lambda-6$. By Caley-Hamilton theorem, we have 
\begin{equation}
A^3-6A^2+11A-6I=0_{3\times3}
\end{equation}
or equivalently
\begin{equation}
A^{-1}=\frac{1}{6}(A^2-6A+11I)
\end{equation}

If we have a system $Ax=b$, where $A$ is an invertibel matrix, then

\begin{align}
x&=A^{-1}b\\
&=\frac{1}{6}(A^2-6A+11I)b\\
&=\frac{1}{6}(A^2b-6Ab+11b)\\
&=L(A^2b,Ab,b)\\
&=K_3(A,b)
\end{align}


\section{Conjugate Gradient Iteration}
\begin{theorem}
Let the conjugate gradient iteration be applied to symetric positive definite matrix problem $Ax=b$. As long as the iteration has not yet converged ($r_n \neq 0$), the algorithm proceeds without division by zero, and we have the following identities of subspaces:
\begin{align}
K_n&=<x_1,x_2,\ldots,x_n>=<p_0,p_1,\ldots,p_{n-1}>\\
&=<r_0,r_1,\ldots,r_{n-1}>=<b,Ab,\ldots,A^{n-1}b>
\end{align}
Moreover, residuals are orthogonal: $r_b^Tr_j=0$, $(j<n)$
and the search directions are A-conjugate: $p_n^TAP_j=0$,$(j<n)$.
\end{theorem}

\end{document}